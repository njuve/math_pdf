\documentclass{ujarticle}[12pt,a4pepar]
\usepackage{geometry}
\usepackage{amssymb}
\usepackage{amsmath}
\usepackage{youngtab}
%\usepackage{titlesec}
\usepackage{amsthm} %newtheorem
\usepackage{url}

%\titleformat*{\section}{\huge\bfseries} %sectionの文字サイズ
%\titleformat*{\subsection}{\huge\bfseries}  %subsectionの文字サイズ

\newtheoremstyle{jplain}% name
{}% space above
{}% space below
{\normalfont}% body font
{}% indent amount
{\bfseries}% theorem head font
{}% punctuation after theorem head
{4pt}% space after theorem head (default: 5pt)
{\thmname{#1}\thmnumber{#2}\thmnote{\hspace{2pt}(#3)}}% theorem head spec

\theoremstyle{jplain}
\newtheorem{dfn}{定義}
\newtheorem{thm}{定理}
\newtheorem{prop}{命題}
\newtheorem{ex}{例}
\newtheorem{cor}{系}
\newtheorem{lem}{補題}
%番号なしver
\newtheorem*{dfn*}{Definition}
\newtheorem*{thm*}{Theorem}
\newtheorem*{prop*}{Proposition}
\newtheorem*{ex*}{Example}
\newtheorem*{cor*}{Corollary}
\newtheorem*{lem*}{Lemma}
\newtheorem*{exe*}{Exercise}
\newtheorem*{remark*}{Remark}
\newtheorem*{q*}{Question}
\newtheorem*{answer*}{Answer}
%\numberwithin{dfn}{section} % 定義番号を「定義2.3」のように印刷
%\numberwithin{thm}{section}
%\numberwithin{prop}{section}

\newcommand\Hom{\operatorname{Hom}} %\Homの定義

\renewcommand\proofname{\bf 証明}
%enumarate環境での番号表示
\renewcommand\labelenumi{(\arabic{enumi})}
\renewcommand\labelenumii{(\alph{enumii})}
\renewcommand\labelenumiii{(\roman{enumiii})}

\title{About Kostka numbers}
\author{}
\date{}

\begin{document}
\maketitle
\section{What is Kostka Numbers}
Kostka numbersは色々なところに出てくるので,色々定義がある.
2通りの定義を書いておく.
\subsection{Semi-standard 盤の数としての定義}
generalised Young tableau の定義から始めて,Kostka数の定義をする.
最後に重複度への応用も書いておく.
\begin{dfn*}
  shape $\lambda$, content $\mu = (\mu_1, \dots, \mu_m)$の\textbf{generalised Young tableau} とは,
  shape $\lambda$ のYoung図形の各boxに,$i \in \mathbb{Z}_{>0}$を$\mu_i$個書き込んだもの.
  特に,行に関してweakly increasing, 列に関してstrictly increasingになっているgeneralised Young tableau
  を\textbf{semi-standard盤}という.
\end{dfn*}

\begin{remark*}
  content $\mu$は,partitionである必要でなく,compositionで十分.
  つまり,$\mu_1 \ge \mu_2 \ge \cdots \ge \mu_m$はいらない.
  例えば,$\mu = (0,2,1)$なども取れる.
\end{remark*}

\begin{ex*}
  $\lambda = (3,2), \mu = (2,2,1)$.
  $\lambda = (3,2)$に$1,1,2,2,3$を埋める.
  \[
    \Yvcentermath1 T_1 = \young({122},{13}),\ T_2 = \young({112},{23})
  \]
  $T_1$は単なるgeneralised Young tableau (列はstrictlyなので), $T_2$はsemi-standard.
\end{ex*}

\begin{dfn*}
  \textbf{Kostka number} $K_{\lambda \mu}$とは,shape$\lambda$, content$\mu$のsemi-standard盤の数.
\end{dfn*}

\begin{ex*}
  $\lambda = (3,2), \mu = (2,2,1)$の半標準盤を列挙すると,
  \[
    \Yvcentermath1 \young({112},{23}),\ \young({113},{22})
  \]
  なので,$K_{\lambda \mu} = 2$.
\end{ex*}
\subsection{Schur多項式の係数としての定義}
  \begin{dfn*}
    shape $\lambda$のsemi-standard盤全体をSSTABと書く.
    このとき,Schur多項式$s_{\lambda}(x_1, \dots, x_k)$は次で定義される.
    \[
      s_{\lambda}(x_1, \dots, x_k) := \sum_{T \in SSTAB(\lambda, k)} x^T
    \]
  \end{dfn*}
  \begin{ex*}
    $\lambda = (2,1)$の半標準盤は,
    \[
      \Yvcentermath1
      \young({11},{2})\ \ \young({11},{3})\ \ \young({12},{2})\ \ \young({12},{3})\ \
      \young({13},{2})\ \ \young({13},{3})\ \ \young({22},{3})\ \ \young({23},{3})
    \]
    で,これに対応して,
    \[
      x_1^2x_2\ \ x_1^2x_3\ \ x_1x_2^2\ \ x_1x_2x_3\ \ x_1x_3x_2\ \ x_1x_3^2\ \ x_2^2x_3\ \ x_2x_3^2
    \]
    が定まり,
    \[
      s_{\lambda}(x_1,x_2,x_3) = x_1^2x_2 + x_1^2x_3 + x_1x_2^2 + 2 x_1x_2x_3 + x_1x_3^2 + x_2^2x_3 + x_2x_3^2
    \]
    となる.
  \end{ex*}

  \begin{prop*}
    $s_{\lambda} = \sum_\beta K_{\lambda \beta} x^\beta$
  \end{prop*}

  % \begin{prop*}
  %     $\lambda \triangleleft \mu$とする.このとき,
  %       \[
  %         dim Hom(M^{\lambda}, S^{\mu}) = \text{(shape $\lambda$, content $\mu$のsemi-standard盤の数)}
  %       \]
  % \end{prop*}
\subsection{応用}
  \begin{prop*}
    $M^{\lambda} = \bigoplus_{\lambda \triangleleft \mu} K_{\lambda \mu}S^{\mu}$
  \end{prop*}

\section{Survey}
\subsection{Wikipedia}
\begin{quote}
  In general, there are no nice formulas known for the Kostka numbers.
\end{quote}
nice formula?
\subsection{古典}
それっぽいことは書いてなさそう,ただのメモ.無視していい.
\subsubsection{sagan}
\[
  K_{\lambda \mu} \neq 0 \Rightarrow \lambda \triangleright \mu
\]

\subsubsection{Fulton}

\begin{quotation}
  \begin{exe*}
    If $r = (r_1, \cdots, r_n)$ and $c = (c_1, \cdots, c_n)$, show that the number of
    m by n matrices with nonnegative integers entries and row sums $r_1, \cdots, r_n$
    and columns sums $c_1, \cdots, c_n$ is $\sum K_{\lambda r} K_{\lambda c}$, the sums
    taken over all partitions $\lambda$ of $\sum r_i = \sum c_j$, where $K_{\lambda r}$
    and $K_{\lambda c}$ are costka numbers.
    Show that the number of symmetric n by n matrices with nonnegative integer entries
    and row sums $r_1, \cdots, r_n$ is $\sum K_{\lambda r}$, the sum over all partitions
    $lambda$ of $\sum r_i$
  \end{exe*}
\end{quotation}

\subsection{StackExchange}
Kostka numbresについてのsurveyは見つからない.
組合せ論的表現論とか組合せ論のsurveyに少し書かれていたりするのかもしれない.
現状Kostka数についてのsurvey的なものとしては,次の質問と回答がいい.
回答者はUCLAの教員.リンクが豊富なのでwebで見たほうがいい.
\begin{q*}
  Why is a general formula for Kostka numbers "unlikely" to exist?

  In reference to Stanley's Enumerative Combinatorics Vol. 2: right after he has
  defined Kostka numbers (section 7.10), he mentions that it is unlikely that a
  general formula for $K_{\lambda \mu}$ exists, where $K_{\lambda \mu}$ is the
  number of semistandard Young tableaux of shape $\lambda$ and type $\mu$ with
  $\lambda \vdash n$ and $\mu$ a weak composition of $n$.
  Why? In particular, is this an expression of something rigorous, and if so, what?
\end{q*}

\begin{answer*}
  This is a really good question, the kind of question I think about from time to time.
  The problem with this question is that it is so much imprecise,
  it is basically open ended. Here are some variations on the way to make question precise.

  1) By a "general formula" you mean a product of some kind of factorials.
  This is rather uninteresting, since it's unclear what those factorials would be.
  Kostka numbers tend to be chaotic,
  so I am sure you can find relatively small partitions with annoying large prime factors.
  What do you do next? One can also ask about asymptotic results which don't allow this,
  in the flavor of de Bruijn (see Section 6).
  But again there are too many choices to consider, and none are really enlightening.

  2) There is a formal notion of \#P-completness,
  a computational complexity class, loosely corresponding to hard counting problems (see WP).
  It is known that Kostka numbers are \#P-complete (see this paper).
  This means that computing Kostka numbers in full generality is just as hard as computing
  the number of 3-colorings in graphs. 3SAT solutions, etc.

  3) Continuing with the theme "formula" as a polynomial algorithm. Such "formulas"
  do exist in special cases then. For example, if the number of rows in both partitions
  is fixed, Kostka numbers become the number of integer points in a finite dimensional
  polytope (see e.g. this nice presentation), which can be computed in polynomial time
  (see this book).

  4) Alternatively, there is a rather weak notion of "formula" due to Wilf
  (see here, by subscription). Roughly, he asks for the algorithm which is asymptotically
  faster than trivial enumeration. But then one can use the "inverse Kostka numbers" which
  have their own combinatorial interpretation (see here), which are similar but perhaps slightly
  faster to compute. Since Wilf only asks for a little better than trivial bound, one can compute
  the whole matrix of Kostka numbers which has sub-exponential size $p(n)$, while Kostka numbers are
  exponential under mild conditions.

Hope this helps.

shareciteimprove this answer

\end{answer*}
\url{https://math.stackexchange.com/questions/17891 - } \\
\url{/why-is-a-general-formula-for-kostka-numbers-unlikely-to-exist}


\subsection{papers}
\subsubsection{A DETERMINANT-LIKE FORMULA FOR THE KOSTKA NUMBERS}
\label{formula}
これはそれっぽい,ちゃんと読んでないから全然わからない.2005.
\begin{quotation}
  \begin{thm*}
    The number $K_{\alpha,\beta}$ of semistandard Young tableaux of shape $\alpha$
    and of content $\beta$ is given by
    $K_{\alpha,\beta} = \sum sgn(\sigma)\mu_{\beta}(\alpha - (k) + (\sigma(k))$.
  \end{thm*}
\end{quotation}

\subsubsection{AN ELEMENTARY METHOD FOR COMPUTING THE KOSTKA COEFFICIENTS}
初等的な方法(連立方程式・不等式を解く)で,いくつかの場合に具体的に計算し,公式を導出してる.
各場合で同一の公式が出ているよう.ただ参考文献が少ない.SaganとFultonだけ.2010.
\begin{quotation}
  Abstract.
  We present a simple and elementary method for com- puting the Kostka numbers.
  We use this method to give compact formulas for Kπ,$\mu$ in some special cases.
\end{quotation}

\subsubsection{On the complexity of computing Kostka numbers and Littlewood-Richardson coefficients}
"Thus, unless P = NP, which is widely disbelieved,
there do not exist efficient algorithms that compute these numbers."
\begin{quotation}
  Abstract.
  Kostka numbers and Littlewood-Richardson coefficients appear in combinatorics and representation theory.
  Interest in their computation stems from the fact that they are present in quantum mechanical computations since Wigner [15].
  In recent times, there have been a number of algorithms proposed to perform this task [1–3, 11, 12].
  The issue of their computational complexity has received at-tention in the past, and was raised recently by E. Rassart in [11].
  We prove that the problem of computing either quantity is \#P-complete. Thus, unless P = NP, which is widely disbelieved,
  there do not exist efficient algorithms that compute these numbers.
\end{quotation}

\subsubsection{Kostka Numbers and Littlewood-Richardson Coefficients: Distributed Computation}
"Computing Kostka numbers and Littlewood-Richardson coefficients remains of great interest in combinatorics"
$K_{N\lambda N\mu}$に関する公式,hiveを使った計算,maple使用.perspectiveの節がある.
\begin{quotation}
  Abstract. Computing Kostka numbers and Littlewood-Richardson coefficients remains of great interest in combinatorics,
  and Brian G. Wybourne was among the first people to design software -SCHUR- for their computation.
  The efficiency of existing software -SCHUR, Stembridge package, LattE,
  Cochet’s programs- is generally constrained by the lengths or weights of partitions.
  This work describes another method, based on the hives model,
  applying distributed computing tech- niques to the determination of generating
  polynomials for stretched Kostka numbers and stretched Littlewood-Richardson coefficients.
  This method can be used to ”quickly” find such polynomials, with the help (of a predefined subset) of
  the available computers of the Local Area Network.
\end{quotation}
時期が分からないけど少なくとも2005年以降.

\subsubsection{On a Formula for the Kostka Numbers}
\begin{quotation}
  Abstract. From Kostant’s multiplicity formula for general linear groups,
  one can derive a formula for the Kostka numbers.
  In this note we give a combinatorial proof of this formula.
\end{quotation}
2006.上の\ref{formula}の別証明.

\subsubsection{STRETCHED LITTLEWOOD-RICHARDSON AND COEFFICIENTS}
$K_{N\lambda N\mu}$に関する公式.
$P(N) = K_{N\lambda N\mu}$となる多項式$P$を見つける問題は,特別な場合に限っては,示されている.
conjectureも一つ乗ってる.

\subsubsection{ Polynomiality properties of the Kostka numbers and Littlewood-Richardson coefficients}
StackExchangeで"nice presentation"と紹介されているやつ,open problemも一つ書いている.
ある分割のkostka数は,格子点を数える問題に帰着するような話.
$sl$の表現論が出てきたりする.
面白そうだけどこれだけだとよくわからないとこが結構ある.
この論文はLittlewood-Richardson coefficientsの場合にだけ書かれている.
hiveという比較的新しい組合せ論の対象を使って求められるらしい.

\subsubsection{COMBINATORIAL REPRESENTATION THEORY}
組合せ論的表現論に関するsurvey.
1997年で少し古いけど包括的.

\subsubsection{The Ubiquitous Young Tableau}
ヤング図形に関するsurvey.
1988年,ヤング図形が登場する分野を網羅してる.
リトルウッドリチャードソン係数は色々な場面で出て来る$\cdots$みたいな文脈でよく参考文献にされている.

\subsubsection{REPRESENTATION THEORY OF SYMMETRIC GROUPS AND RELATED HECKE ALGEBRAS}
対称群の表現論に関連する話題のsurvey.2009年,70ページくらいの大作.

\end{document}
