\documentclass[dvipdfmx,autodetect-engine]{article}
%---------------------package
\usepackage{geometry}
\usepackage{amssymb}
\usepackage{amsmath}
\usepackage{amsthm}
\usepackage{tikz}
\usetikzlibrary{positioning}

% ----------------------------- 
\theoremstyle{definition}
\newtheorem{Def}{定義}
\newtheorem{Th}{定理}
\newtheorem{Prop}{命題}
\newtheorem{Ex}{例}
\newtheorem{Cor}{系}
\newtheorem{Lem}{補題}
\newtheorem{Prob}{問}

% ------------------------- enumareteで(1)みたいにする
\renewcommand\labelenumi{(\arabic{enumi})}
\renewcommand\labelenumii{(\alph{enumii})}
\renewcommand\labelenumiii{(\roman{enumiii})}

%----------------------------命令定義
\DeclareMathOperator{\Hom}{Hom}
\DeclareMathOperator{\End}{End}
\DeclareMathOperator{\GL}{GL}
\DeclareMathOperator{\gl}{gl}
\DeclareMathOperator{\sllie}{sl}
\DeclareMathOperator{\sublie}{\mathfrak{a}}
\DeclareMathOperator{\lie}{\mathfrak{g}}
\DeclareMathOperator{\cartan}{\mathfrak{h}}
\DeclareMathOperator{\U}{U}


%---------------------------make title
\title{}
\author{}
\date{}
%-----------------------------document
\begin{document}
% \maketitle
\section{$A^{(1)}_{r-1}$型の量子群}
    この章ではカッツ・ムーディリー環を定義して,少し基本的なことをやった後で,$A^{(1)_{r-1}}$型の量子群を定義します.
\subsection{カッツ・ムーディリー環}
    基本的なアイデアはカルタン行列からリー環を構成すること.
    \begin{Def}
        正方行列$A = (a_{ij}) \quad (a_{ij} \in \mathbb{Z})$が
        \[
            \begin{cases}
                a_{ii} = 2 \\
                a_{ij} \leq 0 \quad (i \neq j) \\
                a_{ij} = 0 \iff a_{ji} = 0
            \end{cases}
        \]
        を満たす時,(一般)カルタン行列という.
    \end{Def}
    
    \begin{Def}
        カルタン行列$A = (a_{ij})_{i \leq i \leq r}$の実現とは,
        ベクトル空間$\mathfrak{h}$と一次独立な元の集合
        $\Pi=\left\{\alpha_{i}\right\}_{0 \leq i<r} \subset \mathfrak{h}$, 
        $\Pi^{v}=\left\{h_{j}\right\}_{0 \leq j<r} \subset \mathfrak{h}^{*}$
        の3つの組$(\mathfrak{H}, \Pi, \Pi^{v})$
        であって,
        \[
            \dim\mathfrak{h} = 2r - rankA,\quad \alpha_j(h_i) = a_{ij}
        \]
        を満たすものをいう.
    \end{Def}
    
    
 \end{document}
