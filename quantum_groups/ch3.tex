\documentclass[dvipdfmx,autodetect-engine]{article}
%---------------------package
\usepackage{geometry}
\usepackage{amssymb}
\usepackage{amsmath}
\usepackage{amsthm}
\usepackage{tikz}
\usetikzlibrary{positioning}

% ----------------------------- 
\theoremstyle{definition}
\newtheorem{Def}{定義}
\newtheorem{Th}{定理}
\newtheorem{Prop}{命題}
\newtheorem{Ex}{例}
\newtheorem{Cor}{系}
\newtheorem{Lem}{補題}
\newtheorem{Prob}{問}

% ------------------------- enumareteで(1)みたいにする
\renewcommand\labelenumi{(\arabic{enumi})}
\renewcommand\labelenumii{(\alph{enumii})}
\renewcommand\labelenumiii{(\roman{enumiii})}

%----------------------------命令定義
\DeclareMathOperator{\Hom}{Hom}
\DeclareMathOperator{\End}{End}
\DeclareMathOperator{\GL}{GL}
\DeclareMathOperator{\gl}{gl}
\DeclareMathOperator{\sllie}{sl}
\DeclareMathOperator{\sublie}{\mathfrak{a}}
\DeclareMathOperator{\lie}{\mathfrak{g}}
\DeclareMathOperator{\cartan}{\mathfrak{h}}
\DeclareMathOperator{\U}{U}


%---------------------------make title
\title{}
\author{}
\date{}
%-----------------------------document
\begin{document}
% \maketitle
\section{$A^{(1)}_{r-1}$型の量子群}
    この章ではカッツ・ムーディリー環を定義して,少し基本的なことをやった後で,$A^{(1)}_{r-1}$型の量子群を定義します.
    まず.このノートで扱う$A^{(1)}_{r-1}$型とは,$\sllie$の変形.
    \[
        L = \sllie(n, \mathbb{C}) \oplus \mathbb{C}[t, t^{-1}] \oplus \mathbb{C}c \oplus \mathbb{C}d
    \]
    ここでリー積は
    \begin{align*}
        \left[X \otimes t^{n}, Y \otimes t^{m}\right]
        &=[X, Y] \otimes t^{n+m}+n(t r X Y) \delta_{0, n+m} c\\
        \left[c, X \otimes t^{n}\right]
        &=0 \\
        \left[d, X \otimes t^{n}\right]
        &=n X \otimes t^{n}, \\
        [c, d]
        &=0
    \end{align*}
    これの生成元と基本関係による表示のためにKac-Moodyリー環をまずやります.
\subsection{カッツ・ムーディリー環}
    基本的なアイデアはカルタン行列からリー環を構成すること.
    \begin{Def}
        正方行列$A = (a_{ij}) \quad (a_{ij} \in \mathbb{Z})$が
        \[
            \begin{cases}
                a_{ii} = 2 \\
                a_{ij} \leq 0 \quad (i \neq j) \\
                a_{ij} = 0 \iff a_{ji} = 0
            \end{cases}
        \]
        を満たす時,(一般)カルタン行列という.
    \end{Def}
    
    \begin{Def}
        カルタン行列$A = (a_{ij})_{i \leq i \leq r}$の実現$(\mathfrak{h}, \Pi, \Pi^{v})$$\overset{def}{\iff}$,
        \begin{enumerate}
            \item $\mathfrak{h}$:ベクトル空間
            \item $\Pi=\left\{\alpha_{i}\right\}_{0 \leq i<r} \subset \mathfrak{h}$:一次独立
            \item $\Pi^{v}=\left\{h_{j}\right\}_{0 \leq j<r} \subset \mathfrak{h}^{*}$:一次独立
            \item $\dim\mathfrak{h} = 2r - rankA$
            \item $\alpha_j(h_i) = a_{ij}$
        \end{enumerate}
    \end{Def}
    
    このノートで考えるのは,$A^{(1)}_{r-1}$型のカルタン行列.
    例えば,
    \begin{Ex}
        $r = 2$:
        \[
            \begin{pmatrix}
                2  & -2 \\
                -2 & 2
            \end{pmatrix}
        \]
        $r \geq 3$:
        \[
            \begin{pmatrix}
                2  & -1 & \cdots & \cdots & -1\\
                -1 & 2  & -1 & \cdots & -1\\
                \cdots & \cdots  & \cdots & \cdots & \cdots\\
                -1 & \cdots  & \cdots & -1 & 2\\
            \end{pmatrix}
        \]
        $r \geq 3$の時,このカルタン行列の実現を考える.
        \begin{align*}
            \mathfrak{h} &= (\bigoplus \mathbb{C}h_i) \oplus \mathbb{C}d\\
            \mathfrak{h}^* &= (\bigoplus \mathbb{C}\alpha_i) \oplus \mathbb{C}\Lambda_0\\
            \Pi &= \left\{\alpha_{i}\right\}_{0 \leq i \leq n-1}\\ 
            \Pi^{\vee} &= \left\{h_{j}\right\}_{0 \leq j \leq r-1}
        \end{align*}
        ただし,ここで$\alpha_i$と$h_i$は次を満たす$\mathfrak{h}$と$\mathfrak{h}^*$の基底.
        \begin{align*}
            \alpha_{i}\left(h_{j}\right)=
            \left\{
                \begin{array}{cc}{2} & {(i=j)} \\ 
                    {-1} & {(i-j \equiv \pm 1(\bmod r))} \\ 
                    {0} & {(\text{else})}
                \end{array}\right\\
            \alpha_{i}(d)=\delta_{i 0}, \quad \Lambda_{0}\left(h_{j}\right)=\delta_{0 j}, \quad 
            \Lambda_{0}(d)=0
        \end{align*}
    \end{Ex}
    
    \begin{Def}
        $\Tilde{L}(V)$を次の生成元と基本関係で定義されるリー環とする.
        
        生成元:
        \[
            (\bigoplus_{i = 0}^{r-1} \mathbb{C}e_i) \oplus \mathfrak{h} \oplus (\bigoplus_{i = 0}^{r-1} \mathbb{C}f_i)
        \]
        
        基本関係:
        \begin{align*}
            \left[h, e_{i}\right]&=\alpha_{i}(h) e_{i} \\
            \left[h, f_{i}\right]&=-\alpha_{i}(h) f_{i} \\
            \left[e_{i}, f_{j}\right]&=\delta_{i j} h_{i} \\
            \left[h, h^{\prime}\right]&=0 \quad \left(h, h^{\prime} \in \mathfrak{h}\right)
        \end{align*}
        
        $\mathcal{R}_{max}$をイデアル$\Tilde{L}(V) \supset \mathcal{R}$で$\mathfrak{h}\cap\mathcal{R} = 0$を満たすもののうち最大のものとする.
        
        この時,$\Tilde{L}(V)/\mathcal{R}_{max}$をカルタン行列$A$の定義するKac-Moodyリー環と呼び,$L(A)$で表す.
        
        $\Tilde{L}(V)$や$L(A)$の$ad(\mathfrak{h})$に関する非ゼロ同時固有値をルートと呼び,同時固有分解をルート分解と呼ぶ.
    \end{Def}
    
    \begin{Th}
        簡単のため,$A^{(1)_{r-1}}$型$(r \geq 3)$を考える.
        $L(A)$は次の生成元と基本関係で定義されるリー環.
        
        生成元:
        \[
            (\bigoplus_{i = 0}^{r-1} \mathbb{C}e_i) \oplus \mathfrak{h} \oplus (\bigoplus_{i = 0}^{r-1} \mathbb{C}f_i)
        \]
        
        基本関係:
        \begin{align*}
            \left[h, e_{i}\right]&=\alpha_{i}(h) e_{i} \\
            \left[h, f_{i}\right]&=-\alpha_{i}(h) f_{i} \\
            \left[e_{i}, f_{j}\right]&=\delta_{i j} h_{i} \\
            \left[h, h^{\prime}\right]&=0 \quad \left(h, h^{\prime} \in \mathfrak{h}\right)\\
            [e_i[e_i, e_j]] &= 0 \quad (i - j \mod 1)\\
            [e_i, e_j] &= 0 \quad (\text{else})\\
            [f_i[f_i, f_j]] &= 0 \quad (i - j \mod 1)\\
            [f_i, f_j] &= 0 \quad (\text{else})\\
        \end{align*}
    \end{Th}
    
    \begin{Prop}
        $L(A)$のなかで上の関係式は全て成立.
    \end{Prop}
    
    \begin{Prop}
        \begin{enumerate}
            \item
                \begin{align*}
                    A&: \text{カルタン行列}\\
                    &L(A)\\
                    V_+ &= \bigoplus_{i=1}^{r-1}\mathfrak{C}e_i\\
                    V_- &= \bigoplus_{i=1}^{r-1}\mathfrak{C}f_i\\
                    L(V)&:=(V := V_+ \oplus \mathfrak{h} \oplus V_- \text{で生成される自由リー環})\\
                    &\Tilde{L}(V)
                \end{align*}
                この時,三角分解が成立.
                \[
                    \Tilde{L}(V) = L(V_+) \oplus \mathfrak{h} \oplus L(V_-)
                \]
                各直和成分はルート分解を持ち,各ルートは,
                $L(V_+)$上では
                    $\left(\oplus Z_{ \geq 0} \alpha_{i}\right) \backslash\{0\}$の元.
                $\mathfrak{h}$上では
                    $0$.
                $L(V_-)$上では
                    $\left(\oplus Z_{ \leq 0} \alpha_{i}\right) \backslash\{0\}$の元.    
            
            \item $\mathcal{R}_{max}$は存在して一意.
            \item 
                $\mathcal{R}_{\pm} := \mathcal{R}_{max} \cap V_{\pm}$とすると, 
                \[
                    g(A)=\left(g\left(V_{+}\right) / R_{+}\right) \oplus h \oplus\left(g\left(V_{-}\right) / R_{-}\right)
                \]
            \item
                $L(A)$のルート分解を考えると,同時固有値が$0$になるのは$\mathcal{h}$の元に限る.また
                \begin{align*}
                    \Delta &:= \text{ルート系}\\
                    \Delta_+ &:= \Delta \cap \left(\oplus Z_{ \geq 0} \alpha_{i}\right)\\
                    \Delta_- &:= \Delta \cap \left(\oplus Z_{ \leq 0} \alpha_{i}\right)\\
                \end{align*}
                とすると,
                \[
                    \Delta = \Delta_+ \cup \Delta_-
                \]
        \end{enumerate}
    \end{Prop}
    
    \begin{Prop}
       $\exists (L \supset \mathfrak{h}, \{e_i, f_i\}, \Pi = \{\alpha_i\} \subset \mathfrak{h}^*, \Pi^{v} = \{h_i\} \subset \mathfrak{h})$ s.t.,
       \begin{enumerate}
           \item $L = \langle e_{0}, f_{0}, \dots, e_{r-1}, f_{r-1}, \mathfrak{h}\rangle$
           \item 
                \begin{align*}
                    \left[e_{i}, f_{j}\right]&=\delta_{i j} h_{i},\\
                    \left[h, e_{i}\right]&=\alpha_{i}(h) e_{i},\\
                    \left[h, f_{i}\right]&=-\alpha_{i}(h) f_{i}
                \end{align*}
            \item $(\mathfrak{h}, \Pi, \Pi^{v})$はカルタン行列$A$の実現.
            \item $L$の中で$\mathcal{R}\cap\mathfrak{h} = 0$となるイデアルは$\mathcal{R} = 0$
       \end{enumerate}
       この時,$L$はカルタン行列$A$から定まるKac-Moddyリー環$L(A)$と同型.
    \end{Prop}
    
\section{$A^{(1)}_{r-1}$型の量子群}
    次の生成元と基本関係で定義される$\mathbb{Q}(v)$上の単位的結合環を$A^{(1)}_{r-1}\, (r \geq 3)$型の量子群という

    生成元:
        \[
            t_i^{\pm1}, v^{\pm1}, f_i, e_i
        \]
        
        基本関係:
            \begin{align*}
                    t_{i} e_{j} t_{i}^{-1}
                    =v^{\alpha_{j}\left(h_{i}\right)} e_{j}
                    &,\quad 
                    t_{i} f_{j} t_{i}^{-1}
                    =v^{-\alpha_{j}\left(h_{i}\right)} f_{j}\\
                    v^{d} e_{j} v^{d}^{-1}
                    =v^{\delta_{0j}} e_{j}
                    &,\quad 
                    v_{d} f_{j} v_{d}^{-1}
                    =v^{-\delta_{0j}} f_{j}\\
                    \left[e_{i}, f_{j}\right]
                    &=\delta_{i j}
                    \frac{t_{i}-t_{i}^{-1}}{v-v^{-1}}\\
                    \left[v^{d}, t_{i}\right]
                    = 0
                    &, \quad
                    [t_{i}, t_{j}] 
                    = 0,\\
                    v_{d} v_{d}^{-1}
                    = v_{d}^{-1} v_{d}
                    = 1
                    &, \quad
                    t_{i} t_{i}^{-1}
                    = t_{i}^{-1} t_{i}
                    = 1\\
                    {e_{i}^{2} e_{j}
                    -\left(v+v^{-1}\right) e_{i} e_{j} e_{i}
                    &+e_{j} e_{i}^{2}
                    = 0,(i-j=\pm 1)} \\
                    {e_{i} e_{j}
                    &=
                    e_{j} e_{i}(\text{else})}\\
                    {f_{i}^{2} f_{j}
                    -(v+v^{-1}) f_{i} f_{j} f_{i}
                    &+f_{j} f_{i}^{2}
                    = 0,(i-j=\pm 1)} \\
                    {f_{i} f_{j}
                    &=f_{j} f_{i}(\text{else})}
            \end{align*}
            $r = 2$の時は,セールの関係式が
            \begin{align*}
                e_{i}^{3} e_{j}-\left(v^{2}+1+v^{-2}\right) e_{i}^{2} e_{j} e_{i}+\left(v^{2}+1+v^{-2}\right) e_{i} e_{j} e_{i}^{2}-e_{i} e_{i}^{3}=0(i \neq j)\\
                f_{i}^{3} f_{j}-\left(v^{2}+1+v^{-2}\right) f_{i}^{2} f_{j} f_{i}+\left(v^{2}+1+v^{-2}\right) f_{i} f_{j} f_{i}^{2}-f_{j} f_{i}^{3}=0(i \neq j)
            \end{align*}
 \end{document}
