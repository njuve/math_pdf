\documentclass[dvipdfmx,autodetect-engine]{article}
%---------------------package
\usepackage{geometry}
\usepackage{amssymb}
\usepackage{amsmath}
\usepackage{amsthm}
\usepackage{tikz}
\usetikzlibrary{positioning}

% ----------------------------- 
\theoremstyle{definition}
\newtheorem{Def}{定義}
\newtheorem{Th}{定理}
\newtheorem{Prop}{命題}
\newtheorem{Ex}{例}
\newtheorem{Cor}{系}
\newtheorem{Lem}{補題}
\newtheorem{Prob}{問}

% ------------------------- enumareteで(1)みたいにする
\renewcommand\labelenumi{(\arabic{enumi})}
\renewcommand\labelenumii{(\alph{enumii})}
\renewcommand\labelenumiii{(\roman{enumiii})}

%----------------------------命令定義
\DeclareMathOperator{\Hom}{Hom}
\DeclareMathOperator{\End}{End}
\DeclareMathOperator{\GL}{GL}
\DeclareMathOperator{\gl}{gl}
\DeclareMathOperator{\sllie}{sl}
\DeclareMathOperator{\sublie}{\mathfrak{a}}
\DeclareMathOperator{\lie}{\mathfrak{g}}
\DeclareMathOperator{\cartan}{\mathfrak{h}}
\DeclareMathOperator{\U}{U}


%---------------------------make title
\title{}
\author{}
\date{}
%-----------------------------document
\begin{document}
% \maketitle
\section{局所可積分加群と結晶基底}
    4-6章は対称化可能カルタン行列に対して成立するが, $A^{(1)}_{r-1}$型と思って進めます.
    量子群を考えるときの体は$K = \mathbb{Q}(v)$.
    
    この章の目標は結晶基底を定義することです.その為の当面の目標は,柏原作用素を定義することです.
    
\subsection{局所可積分加群}
    量子群はHopf代数です.余積は次のようになっています.
    \begin{Lem}
        $\Delta:U_{v} \to U_{v} \otimes U_{v}$を
        \[
            \begin{array}{c}
                {\Delta\left(t_{i}\right)=t_{i} \otimes t_{i}, \quad \Delta\left(v^{d}\right)=v^{d} \otimes v^{d}} \\ {\Delta\left(e_{i}\right)=1 \otimes e_{i}+e_{i} \otimes t_{i}^{-1}} \\ 
                {\Delta\left(f_{i}\right)=f_{i} \otimes 1+t_{i} \otimes f_{i}}
            \end{array}
        \]
        と定義するとこれは環準同型でwell-defined.
    \end{Lem}
    
    この余積で量子群のテンソル積表現が定義される: 
    
    $M, N:$ $U_{v}$-加群,$x \in U_{v}$
    \[
        x \cdot (M \otimes N) := \Delta(x)(M \otimes N)
    \]
    
    \begin{Def}
        $U_{v}$-加群$M$が局所可積分加群:
        \begin{enumerate}
            \item 
                \label{def_int_weight}
                $M$はウェイト加群をもつ
                \begin{align*}
                        P &= \{ \lambda \in H^{*} \mid \lambda(h_i), \lambda(d) \in \mathbb{Z}\}\\
                        M_{\lambda} 
                        &= \{ 
                            m \in M 
                            \mid
                            t_im = v^{\lambda(h_i)}m\,
                            (0 \leq i \leq r),
                            v^{d}m  = v^{\lambda(d)}m\},\\
                        M &= \bigoplus_{\lambda \in P} M_{\lambda}
                \end{align*}
            \item 
                \label{def_int_weight_dim}
                $\dim M_{\lambda} \leq \infty$
            \item 
                \label{def_int_nil}
                $U_{v}(L_i) := \langle e_i, f_i, h_i \rangle \subset U_{v}$に対し,$\each m \in M$は有限次元$U_{v}(L_i)$-部分加群に含まれる.
                $(\iff \dim U_{v}m \leq \infty)$
        \end{enumerate}
    \end{Def}
    (\ref{def_int_nil})から(\ref{def_int_nil}'): $e_{i}^{k} = f_{i}^{l} = 0,\,(\exists k, l \geq 0)$がわかる .
    また(\ref{def_int_weight})と(\ref{def_int_nil}')から(\ref{def_int_nil})がわかる.
    
    大切な例は最高ウェイト加群(これは後で).
    \begin{Ex}
        $r \geq 2$, $M=K^{2} \otimes K\left[\xi, \xi^{-1}\right]$とする.
        $U_{v}$からの作用を次のように定義すると$M$は局所可積分加群.
        \begin{align*}
            e_{0}=
                \left[ 
                    \begin{array}{ll}
                        {0} & {0} \\ 
                        {1} & {0}
                    \end{array}
                \right]
                \otimes
                \xi,
            &\quad
            f_{0}=
                \left[
                    \begin{array}{ll}
                        {0} & {1} \\
                        {0} & {0}
                    \end{array}
                \right]
                \otimes
                \xi^{-1}\\
            e_{1}=
                \left[ 
                    \begin{array}{ll}
                        {0} & {1} \\ 
                        {0} & {0}
                    \end{array}
                \right]
                \otimes
                1,
            &\quad
            f_{1}=
                \left[
                    \begin{array}{ll}
                        {0} & {0} \\
                        {1} & {0}
                    \end{array}
                \right]
                \otimes
                1\\
            t_{0}=
                \left[
                    \begin{array}{cc}
                        {v^{-1}} & {0} \\
                        {0} & {v}
                    \end{array}
                \right] 
                \otimes
                1, 
            &\quad 
            t_{1}=
                \left[ 
                    \begin{array}{cc}
                        {v} & {0} \\
                        {0} & {v^{-1}}
                    \end{array}
                \right]
                \otimes
                1\\
            v^{d}\left(m \otimes \xi^{j}\right)
            &=v^{j} m \otimes \xi^{j}
        \end{align*}
    \end{Ex}
    \begin{Ex}
        $M$: 局所可積分加群
        $e_i, t_i$の作用を$-e_i, -t_i$に変えると$M$は局所可積分加群でなくなる.
    \end{Ex}
\subsection{柏原作用素の為の準備}
    $U_{v}(\sllie_2)$が基本的な道具になるのでそれをやる.
    まずは$U_{v}(\sllie_2)$の有限次元表現の分類.
    
    \begin{Prop}
        $U_{v}(\sllie_2)$の有限次元表現既約表現の同値類の完全代表系は以下の通り.ただしここで
        \[
            [k] := \frac{v^{k} - v^{-k}}{v - v^{-1}}.
        \]
        
        $V_{l}^{\pm} 
        := \langle m_0^{\pm}, \dots, m_{l-1}^{\pm}\rangle$: 
        \begin{align*}
        e &= \pm
            \begin{bmatrix}
                0 & [l] & \\
                  &\cdots&\cdots&\\
                  &      & 0 & [1]\\
                  &      &   & 0 
            \end{bmatrix}\\
        f &= 
            \begin{bmatrix}
                0 &  & \\
                [l]  &0&&\\
                  & \cdots  & \cdots & \\
                  &      & [l]  & 0 
            \end{bmatrix}\\
        t &= \pm
            \begin{bmatrix}
                v^{l} &   \\
                  & v^{l-2}\\
                  & \cdots  & \cdots & \\
                  &     &   & v^{-l} 
            \end{bmatrix}
        \end{align*}
    \end{Prop}
    \begin{Ex}
       $l = 2$:
       \[
            V_2^{+} = <m_0, m_1, m_2>
       \]
       \begin{align*}
           e = \pm
                &\begin{bmatrix}
                    0 & [2] & 0\\
                    0 &  0  & [1]\\
                    0 &  0  & 0 
                \end{bmatrix}\\
            f = 
                &\begin{bmatrix}
                    0 & 0 & 0\\
                    [2] &0&0\\
                     0 & [1] & 0\\
                \end{bmatrix}\\
            t = \pm
                &\begin{bmatrix}
                    v^{2} & 0 &0  \\
                    0 & v^{0} & 0\\
                    0  & 0 & v^{-2} 
                \end{bmatrix}
        \end{align*}
        \begin{align*}
            (m_0, m_1, m_2)e &= ([2]m_1, [1]m_2, 0)\\
            (m_0, m_1, m_2)f &= (0, [1]m_1, [2]m_2)\\
            (m_0, m_1, m_2)t &= (v^2m_0, v^0m_1, v^{-2}m_2)\\
        \end{align*}
    \end{Ex}
    一般に上の作用は
    \begin{align*}
        \begin{cases}
            em_i = [l-i]m_{i+1}\\
            fm_i = [i]m_{i-1}\\
            tm_i = v^{l-i}m_i
        \end{cases}
    \end{align*}
    ともかける.
    \begin{proof}
        \begin{Lem}
            $V_l$に対し$e,f$の作用は冪零.
        \end{Lem}
    \end{proof}
    
    \begin{Def}
        量子カシミール元$C$:
        \[
            C := fe + \frac{vt - v^{-1}t^{-1}}{(v - v^{-1})^2}
        \]
        $C$は, $e, f, t$と可換.また$V_{l}^{\pm}$上で互いに異なる値
        \[
            \pm \frac{[l+1]}{v - v^{-1}}
        \]
        をとる.
    \end{Def}
\subsection{結晶基底の定義}
\end{document}
